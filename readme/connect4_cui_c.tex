% latex uft-8
\documentclass[uplatex,a4paper,11pt,oneside,openany]{jsbook}
%
\usepackage[dvipdfmx]{graphicx}
\usepackage{amsmath,amssymb}
\usepackage{bm}
\usepackage{graphicx}
\usepackage{ascmac}
\usepackage{setspace}
\usepackage{here}
\usepackage{listings,jlisting} %日本語のコメントアウトをする場合jlistingが必要
%ここからソースコードの表示に関する設定
\usepackage{xcolor,comment}

\definecolor{mygreen}{rgb}{0,0.6,0}
\definecolor{mygray}{rgb}{0.5,0.5,0.5}
\definecolor{mymauve}{rgb}{0.58,0,0.82}

\begin{comment}
\lstset{
  backgroundcolor=\color{white},   % choose the background color; you must add \usepackage{color} or \usepackage{xcolor}; should come as last argument
  basicstyle=\footnotesize,        % the size of the fonts that are used for the code
  breakatwhitespace=false,         % sets if automatic breaks should only happen at whitespace
  breaklines=true,                 % sets automatic line breaking
  captionpos=b,                    % sets the caption-position to bottom
  commentstyle=\color{mygreen},    % comment style
  deletekeywords={...},            % if you want to delete keywords from the given language
  escapeinside={\%*}{*)},          % if you want to add LaTeX within your code
  extendedchars=true,              % lets you use non-ASCII characters; for 8-bits encodings only, does not work with UTF-8
  firstnumber=1000,                % start line enumeration with line 1000
  frame=single,	                   % adds a frame around the code
  keepspaces=true,                 % keeps spaces in text, useful for keeping indentation of code (possibly needs columns=flexible)
  keywordstyle=\color{blue},       % keyword style
  language=Octave,                 % the language of the code
  morekeywords={*,...},            % if you want to add more keywords to the set
  numbers=left,                    % where to put the line-numbers; possible values are (none, left, right)
  numbersep=5pt,                   % how far the line-numbers are from the code
  numberstyle=\tiny\color{mygray}, % the style that is used for the line-numbers
  rulecolor=\color{black},         % if not set, the frame-color may be changed on line-breaks within not-black text (e.g. comments (green here))
  showspaces=false,                % show spaces everywhere adding particular underscores; it overrides 'showstringspaces'
  showstringspaces=false,          % underline spaces within strings only
  showtabs=false,                  % show tabs within strings adding particular underscores
  stepnumber=2,                    % the step between two line-numbers. If it's 1, each line will be numbered
  stringstyle=\color{mymauve},     % string literal style
  tabsize=2,	                   % sets default tabsize to 2 spaces
  title=\lstname                   % show the filename of files included with \lstinputlisting; also try caption instead of title
}
\end{comment}

%ここからソースコードの表示に関する設定

\lstdefinestyle{customc}{
  belowcaptionskip=1\baselineskip,
  breaklines=true,
  numbers=left,
  frame=single,
  xleftmargin=\parindent,
  language=C,
  showstringspaces=false,
  basicstyle=\footnotesize\ttfamily,
  keywordstyle=\bfseries\color{green!40!black},
  commentstyle=\itshape\color{purple!40!black},
  identifierstyle=\color{blue},
  stringstyle=\color{orange},
}

\lstdefinestyle{customjava}{
  belowcaptionskip=1\baselineskip,
  breaklines=true,
  numbers=left,
  frame=single,
  xleftmargin=\parindent,
  language=java,
  showstringspaces=false,
  basicstyle=\footnotesize\ttfamily,
  keywordstyle=\bfseries\color{green!40!black},
  commentstyle=\itshape\color{purple!40!black},
  identifierstyle=\color{blue},
  stringstyle=\color{orange},
}

\lstdefinestyle{customasm}{
  belowcaptionskip=1\baselineskip,
  frame=L,
  xleftmargin=\parindent,
  language=[x86masm]Assembler,
  basicstyle=\footnotesize\ttfamily,
  commentstyle=\itshape\color{purple!40!black},
}

\lstset{escapechar=@,style=customc}

\makeatletter
\def\ps@plainfoot{%
  \let\@mkboth\@gobbletwo
  \let\@oddhead\@empty
  \def\@oddfoot{\normalfont\hfil-- \thepage\ --\hfil}%
  \let\@evenhead\@empty
  \let\@evenfoot\@oddfoot}
  \let\ps@plain\ps@plainfoot
\renewcommand{\chapter}{%
  \if@openright\cleardoublepage\else\clearpage\fi
  \global\@topnum\z@
  \secdef\@chapter\@schapter}
\makeatother
%
\newcommand{\maru}[1]{{\ooalign{%
\hfil\hbox{$\bigcirc$}\hfil\crcr%
\hfil\hbox{#1}\hfil}}}
%
\setlength{\textwidth}{\fullwidth}
\setlength{\textheight}{40\baselineskip}
\addtolength{\textheight}{\topskip}
\setlength{\voffset}{-0.55in}
%
\begin{document}
% START DOCUMENT
%
% COVER
\begin{center}
  \huge \par
  \vspace{15mm}
  \huge \par
  \vspace{15mm}
  \LARGE 連珠 (CUI - C Language) \par
  \vspace{100mm}
  \Large \date \par
  \vspace{15mm}
  \Large \par
  \vspace{10mm}
  \Large S.Matoike \par
  \vspace{10mm}
\end{center}
\thispagestyle{empty}
\clearpage
\addtocounter{page}{-1}
\newpage
\setcounter{tocdepth}{3}
%
\tableofcontents
%

\section{連珠(五目並べ)}

\subsection{定数定義と関数プロトタイプ}

\begin{figure}[H]
  \centering
    \begin{tabular}{c}
\begin{minipage}{1.0\hsize}
  \centering
\begin{lstlisting}[caption=定数定義:連珠(五目並べ),label=prog8]
#include <stdio.h>
#include <stdlib.h>
#include <sysexits.h>
//盤面の広さ設定
#define BOARD_SQUARE 20
//盤面の出力
void Board_Output(void);
//入力処理
void Game(void);
//盤面の調査
void Board_Scan(int x, int y);
//置いた場所を中心に並ぶ個数の調査
int Board_Scan_Sub( int x, int y, int move_x, int move_y );
//ゲーム終了処理
void Finish(void);
int board[BOARD_SQUARE][BOARD_SQUARE] = {{0}};
int player_number = 1;
\end{lstlisting}
\end{minipage}
\end{tabular}
\end{figure}

\subsection{主処理}

\begin{figure}[H]
  \centering
    \begin{tabular}{c}
\begin{minipage}{1.0\hsize}
  \centering
\begin{lstlisting}[caption=主処理:連珠(五目並べ),label=prog8]
int main(void){
    int i;
    Board_Output();
    printf("ゲームスタート!\n");
    for( i = 0; i < (BOARD_SQUARE * BOARD_SQUARE); i++ ){
        Game();
        Board_Output();
        if( player_number < 2 ) player_number++;
        else player_number = 1;
    }
    return 0;
}
\end{lstlisting}
\end{minipage}
\end{tabular}
\end{figure}

\subsection{盤面の表示}

\begin{figure}[H]
  \centering
    \begin{tabular}{c}
\begin{minipage}{1.0\hsize}
  \centering
\begin{lstlisting}[caption=盤面の出力:連珠(五目並べ),label=prog8]
//盤面の出力----------------------------
void Board_Output(void){
    int i, j;
    printf("  ");
    for( i = 0; i < BOARD_SQUARE; i++ ){
        printf("%2d",i);
    }
    puts("");
    for( i = 0; i < BOARD_SQUARE; i++ ){
        printf("%2d",i);
        for( j = 0; j < BOARD_SQUARE; j++ ){
            switch( board[j][i] ){
                case 0:     printf("・"); break;
                case 1:     printf("○"); break;
                case 2:     printf("●"); break;
            }
        }
        puts("");
    }
    puts("");
}
\end{lstlisting}
\end{minipage}
\end{tabular}
\end{figure}

\subsection{結果の表示}

\begin{figure}[H]
  \centering
    \begin{tabular}{c}
\begin{minipage}{1.0\hsize}
  \centering
\begin{lstlisting}[caption=結果の表示:連珠(五目並べ),label=prog8]
//決着~ゲーム終了-----------------------------
void Finish(void){
    Board_Output();
    printf("%dP(",player_number);
    switch( player_number ){
        case 0:     printf("・"); break;
        case 1:     printf("○"); break;
        case 2:     printf("●"); break;
    }
    printf(")の勝利です!\n");

    exit(0);
}
\end{lstlisting}
\end{minipage}
\end{tabular}
\end{figure}

\subsection{手の入力}

\begin{figure}[H]
  \centering
    \begin{tabular}{c}
\begin{minipage}{1.0\hsize}
  \centering
\begin{lstlisting}[caption=入力処理:連珠(五目並べ),label=prog8]
//入力処理-----------------------------
void Game(void){
    int x, y;
    printf("%dP(",player_number);
    switch( player_number ){
        case 0:     printf("・"); break;
        case 1:     printf("○"); break;
        case 2:     printf("●"); break;
    }
    printf(")のターンです。\n");
    while(1){
        while(1){
            printf("置く場所を決めてください(x y) "); scanf("%d %d",&x ,&y);
            if( x >= 0 && x < BOARD_SQUARE && y >= 0 && y < BOARD_SQUARE ) break;
            else printf("その場所には置けません\n");
        }
        if( board[x][y] == 0 ){
            board[x][y] = player_number;
            break;
        } else printf("その場所には置けません\n");
    }
    Board_Scan(x, y);
}
\end{lstlisting}
\end{minipage}
\end{tabular}
\end{figure}

\subsection{勝敗の判定}

\begin{figure}[H]
  \centering
    \begin{tabular}{c}
\begin{minipage}{1.0\hsize}
  \centering
\begin{lstlisting}[caption=盤面チェック:連珠(五目並べ),label=prog8]
//盤面の調査(5個並んだかの調査)--------------------------
void Board_Scan( int x, int y ){
    int n[4];                  //8方向(直線4本分)に並んだ数
    int move_x, move_y;
    int i;
    move_x = 1; move_y = 1;     //[\]方向
    n[0] = Board_Scan_Sub( x, y, move_x, move_y );
    move_x = 0; move_y = 1;     //[│]方向
    n[1] = Board_Scan_Sub( x, y, move_x, move_y );
    move_x = 1; move_y = 0;     //[─]方向
    n[2] = Board_Scan_Sub( x, y, move_x, move_y );
    move_x = -1; move_y = 1;    //[/]方向
    n[3] = Board_Scan_Sub( x, y, move_x, move_y );
    for( i = 0; i < 4; i++ ){
        if(n[i] == 5) Finish();
    }
}

int Board_Scan_Sub( int x, int y, int move_x, int move_y ){
    int n = 1;                                 //置いた場所の1個分で初期化
    int i;
    for( i = 1; i < 5; i++ ){
        if( board[ x+(move_x*i) ][ y+(move_y*i) ]==player_number ) n += 1;
        else break;
    }
    for( i = 1; i < 5; i++ ){
        if( board[ x+(-1*move_x*i) ][ y+(-1*move_y*i) ]==player_number ) n += 1;
        else break;
    }
    return n;
}
\end{lstlisting}
\end{minipage}
\end{tabular}
\end{figure}


%\section*{謝辞}
%\addcontentsline{toc}{chapter}{謝辞}
%
\begin{thebibliography}{99}
  \bibitem{1}
\end{thebibliography}
%
% END DOCUMENT
\end{document}
%
